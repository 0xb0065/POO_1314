\documentclass[10pt,notitlepage]{article}
\usepackage{graphicx} 
\usepackage{verbatim} 
\usepackage[portuguese]{babel} 
\usepackage[utf8]{inputenc}
\usepackage[hmargin=2cm,vmargin=3.5cm,bmargin=2cm]{geometry}


\begin{document}

\subsection{Actividades}
\subsubsection{Classe abstracta \textit{Activity}}

Esta é a classe mais abstracta que contem o conceito de actividade. Contém variáveis comuns a todas as actividades:
\begin{itemize}
\item \textit{String name};
\item \textit{GregorianCalendar date};
\item \textit{double timeSpent};
\item \textit{double calories};
\end{itemize}
tal como os construtores, \textit{getters} e \textit{setters}.






\subsection{Utilizadores}
\subsubsection{Classe abstracta \textit{Person}}
Classe geral para todo tipo de utilizador. As suas variáveis são:
\begin{itemize}
\item \textit{String email} - Email do utilizador;
\item \textit{String password} - Password da conta;
\item \textit{String name} - Nome do utilzador;
\item \textit{char gender} - Género do utilizador;
\item \textit{GregorianCalendar dateOfBirth} - Data de nascimento do utilizador;
\end{itemize}
e contém os métodos construtores \textit{getters} e \textit{setters}

\subsubsection{Classes \textit{User} e \textit{Admin}}

As subclasses de \textit{Person} referem-se a dois possíveis tipos de utilizador; utilizador normal ou utilizador com privilégios de administrador.\\
A classe \textit{Admin} não tem métodos ou variáveis adicionais, visto que este tipo de utilizador apenas opera sobre a base de dados da aplicação.\\
A classe \textit{User} adiciona as seguintes variáveis:
\begin{itemize}
\item \textit{int height};
\item \textit{double weight};
\item \textit{String favoriteActivity};
\item \textit{TreeSet<Activity> userActivities} - Actividades realizadas pelo utilizador;
\item \textit{TreeSet<String> friendsList} - Lista dos amigos do utilizador;
\item \textit{TreeMap<String, ListRecords> records} - Lista dos seus recordes pessoais;
\item \textit{TreeSet<String> messageFriend} - Lista de pedidos de amizade;
\end{itemize}
Respectivos métodos \textit{getters} e \textit{setters}, construtores e métodos auxiliares para gerir os seus amigos, recordes, as suas actividades e estatísticas relevantes. Ainda contém funções auxiliares para a simulação de eventos.

\subsubsection{Comparador}
O tipo \textit{Person} tem apenas um comparador:
\begin{itemize}
\item \textit{ComparePersonByName} - que ordena por ordem alfabética do seu nome.
\end{itemize}


\subsubsection{Statistics}

A classe \textit{Statistics} é usada para mostrar ao utilizador dados relevantes das suas actividades, estes podem ser descriminados por um dado mês ou por um ano. As suas variáveis são:
\begin{itemize}
\item \textit{double timeSpend} ;
\item \textit{double calories};
\item \textit{double distance};
\end{itemize}
contém os respectivos métodos \textit{getters} e \textit{setters} e construtores.



\subsubsection{Classe abstracta \textit{Record}}

Esta classe representa todos os recordes que o utilizador pode bater. Contém apenas uma variável:
\begin{itemize}
\item \textit{String name};
\end{itemize}
métodos construtores, \textit{getName()} e \textit{isEmpty()} que verifica se esse recorde existe ou não.


\subsubsection{\textit{DistancePerTime} e \textit{TimePerDistance}}
Estas classes simboliza os dois diferentes tipos de recordes.\\

\textit{DistancePerTime} é um recorde em que o objectivo é fazer a maior distância para um dado tempo.  
As suas variáveis são:
\begin{itemize}
\item \textit{double recordTime} - Tempo do recorde;
\item \textit{double distance} - Distância registada;
\end{itemize}

Enquanto que \textit{TimePerDistance} representa um recorde de menor tempo para uma certa distância.
As suas variáveis são:
\begin{itemize}
\item \textit{double recordDistance} - Distância do recorde;
\item \textit{double time} - Tempo registado;
\end{itemize}
Estas duas classes têm os mesmos métodos, no entanto os métodos \textit{update()} e \textit{setStatistics()}, estão implementados de maneiras diferentes, tendo em conta que em \textit{DistancePerTime}, quanto maior a distância melhor é o recorde, e no caso do \textit{TimePerDistance},  o melhor recorde é o de menor tempo.


\subsubsection{\textit{ListRecords}}

Classe que agrupa todos os recordes de uma actividade. Tem como variáveis:
\begin{itemize}
\item \textit{String name} - Aqui o nome simboliza o tipo de actividade (Ex: Running, Walking...);
\item \textit{ArrayList$<$Record$>$ recs} - Lista dos recordes;
\end{itemize}
Tem implementado métodos construtores, \textit{getters}, \textit{setters} e ainda um método \textit{updateList()} que aplica a função \textit{update()} a todos os objectos \textit{Record} da lista. (Substitui na lista original caso recorde da segunda lista seja melhor).







\subsection{Eventos}

\subsubsection{Classe abstracta \textit{Event}}
Classe com o conceito mais abstracto de Evento, contém as variáveis:
\begin{itemize}
\item \textit{String name} - Nome do evento;
\item \textit{String tipoActivity} - Tipo de actividade (Running, Walking, ...);
\item \textit{String location} - Onde se realiza a prova;
\item \textit{int maxParticipants} - Número máximo de participantes;
\item \textit{int participants} - Número actual de participantes ;
\item \textit{GregorianCalendar deadline} - Data limite de inscrição;
\item \textit{GregorianCalendar date} - Data de realização;
\item \textit{double duration} - Duração da prova;  $<$--- NÃO ESQUECER DE FALAR DESTA VARIÁVEL
\item \textit{TreeSet$<$User$>$ participantsList} - Lista de participantes;
\item \textit{TreeSet$<$Ranking$>$ ranking} - Classificação dos que acabaram a prova;
\item \textit{TreeSet$<$Ranking$>$ desistentes} - Participantes que desistiram da prova;
\item \textit{TreeSet$<$Simulacao$>$ simula} - Informação relevante para simular cada concorrente;
\end{itemize}
respectivos \textit{getters} e \textit{setters} e os vários construtores. Ainda tem métodos auxiliares para, adicionar um \textit{User}, \textit{Ranking} (desistente ou não) e \textit{Simulacao} aos respectivos \textit{Sets} e para mostrar a classificação geral do evento.


\subsubsection{Tipo de Evento}

Subclasses de Evento (\textit{Marathon}, \textit{HalfMarathon}, \textit{MarathonBTT} e \textit{Trail}), todas estas contem mais uma variável \textit{distance}, que nos casos de \textit{Marathon} e \textit{HalfMarathon} são variáveis \textit{final}, porque este tipo de eventos tem distâncias especificas. Não tem métodos auxiliares para além de \textit{getDistance()}.


\subsubsection{Simulação}

Para guardar dados relevantes à simulação de cada utilizador para um evento, foi criada a classe \textit{Simulacao}. A simulação de cada evento é feita actualizando os dados desta classe a cada km.
 
\begin{itemize}
\item \textit{double tempoGeral} - Tempo acumulado do utilizado na realização da prova.
\item \textit{double tempoMedio} - Tempo médio por km.
\item \textit{int kmDesiste} - Número de km que o utilizador aguenta durante a prova.
\item \textit{User u} - Utilizador associado á simulação. 
\end{itemize}
esta classe, para além dos métodos construtores e \textit{getters} e \textit{setters}, contém apenas um método \textit{actualiza}, que simula a passagem de uma distância (passada como argumento), usando o tempo médio por km e aplicando um factor aleatório (usando \textit{Math.random()}.



\subsubsection{Ranking}

Cada evento, para organizar a sua classificação, utiliza duas colecções de objectos da classe \textit{Ranking}. Uma delas, usada para organizar todos os participantes, que concluíram a prova, por ordem de chegada, a outra onde estão os aqueles que não terminaram, organizados por número de quilómetros realizados.

Esta classe usa as seguintes variáveis:
\begin{itemize}
\item \textit{double time} - Tempo de realizado no evento;
\item \textit{int km} - Número de quilómetros realizados;
\item \textit{User athlete} - Utilizador;
\end{itemize}
Das variáveis \textit{time} e \textit{km}, apenas uma irá ter algum valor para cada utilizador, visto que esta classe é usada para ordenar classificações finais, cada pessoa tem ou um tempo de conclusão do evento ou o número do quilómetro em que desistiu. \textit{Ranking} contém os métodos \textit{getters} e \textit{setters} relevantes, construtores, e para além dos métodos essenciais, foram implementados dois métodos \textit{toString} alternativos, para os dois casos.






    
\section{Considerações finais}

\subsection{Construtores}
Determinadas classes desta aplicação contém construtores que não estão a ser utilizados por nós, foram sendo adicionados enquanto criávamos estas classes, no entanto, mesmo não sendo utilizados, podem ter utilidade futura fora desta aplicação. 

\subsection{Variáveis inutilizadas}
Na classe \textit{Event} a variável \textit{duration} nunca é usada, pois, na criação desta classe pensávamos que íamos precisar da duração do evento, como não demos conta da sua inutilidade até estarmos num estado mais avançado do projecto, decidimos simplesmente ignorar esta variável em vez de a remover. O utilizador nunca terá conhecimento da existência deste problema, pois nunca é pedido esta informação e apenas usamos um construtor que a põe a zero.

\subsection{Actualização de estados}
Durante os testes à funcionalidade da FitnessUM, deparamos com o problema de que, ao ser removida uma actividade, a sua informação continuava a ter relevância para as estatísticas e também para os recordes.
Para melhorar a nossa aplicação foi mudada a implementação das estatísticas para apenas mostrar as actividades que ainda estão na aplicação, em vez de serem actualizadas na adição de uma nova actividade.\\
Para os recordes, no entanto, não foi possível implementar uma funcionalidade equivalente por motivos de tempo, e assim os recordes mostrados ao utilizador são os de todas as suas actividades realizadas, mesmo as que foram removidas.


































\end{document}